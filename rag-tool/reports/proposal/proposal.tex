\documentclass{article}
\usepackage{style}

\begin{document}

\title{RAG EngSci Tool}
\author{Hanhee Lee}
\date{March 10, 2025}
\maketitle  

\section{Introduction}
Engineering Science at the University of Toronto is a one of the hardest undergraduate programs in Canada. 
Specfiically, the program is designed to learn a breadth of courses from all areas of engineering, including civil, software, biomedical, etc in a rigourous manner. 
As a result, I have seen first hand (including myself) many students stuggling with the program, especially those who did not have a strong background in math and physics or enroll in IB/AP programs in high school. 
To help these students, I am proposing a tool based on Retrieval Augmented Generation (RAG) that can help students learn the material in a more efficient manner. Everyone in this day in age is familiar with large language models (LLMs), such as the ones produced by OpenAI and Google. 
However, these models are general-purpose models, and are not tailored to a specific niche. This is why I am proposing a RAG model that is specifically tailored to the Engineering Science program, which will take in knowledge from past exams/midterms, lecture notes of recurring professors, and textbooks used in the program to generate a model that can help students learn the material. 

\section{Objectives}
\begin{itemize}
    \item \textbf{Curate Data:} Collect data from past exams/midterms using web scaping tools. 
    \item \textbf{Train Model:} Train a RAG model using the data collected.
    \item \textbf{Evaluate Model:} Evaluate the model using a variety of metrics, including BLEU score, ROUGE score, etc.
    \item \textbf{Deploy Model:} Deploy the model as a web application that students can use to learn the material.
    \item \textbf{User Study:} Conduct a user study to see if the model is actually helping students learn the material.
\end{itemize}


\section{Background}

\section{Methodology}
\subsection{Data}

\subsection{Evaluation}

\subsection{Model}
RAGFlow is an open-source implmentation of a RAG model that will be used in this project. 

\section{References}
\end{document}